%*****************************************************************************************
%*********************************** First Chapter ***************************************
%*****************************************************************************************

\chapter{Motivation}  %Title of the First Chapter

\ifpdf
    \graphicspath{{Chapter1/Figs/Raster/}{Chapter2/Figs/PDF/}{Chapter2/Figs/}}
\else
    \graphicspath{{Chapter1/Figs/Vector/}{Chapter2/Figs/}}
\fi

Initially, this project was intended to be focused on skin detection and feature recognition for the potential application of hand tracking and posture prediction. However, it was quickly noticed that many authors were sticking to the RGB space, which suffers from disadvantages due to changes in lighting affecting the results. It was also noted that there seemed to be some significant disagreement over the methodology behind using a skin model~\cite{Shin2002a}~\cite{Sigal2000a}~\cite{Skarbek1994}~\cite{Soriano2000a}~\cite{Terrillon1999a}~\cite{Vezhnevets2003}~\cite{Brown2001a}. It appears that the reason many authors avoided luminosity-oriented color spaces was down to the computational intensity of performing the transform, the loss of information, and what seems like a rather perplexing issue to do with white-out, black-out and camera calibration. After reading over the literature, it was decided that two of these three issues were soluble, and the third was worthy of investigation.


The purpose of this project is to create a chromatic skin and feature detector for application in mobile devices. Using a given device's built-in CCD camera, objects with characteristics matching those of human skin are identified. This presents a number of challenges. Since we are trying to use the chromatic information of human skin to distinguish objects (e.g. hands, fingers) in a given scene, working in a chromatic space helps to simplify this problem. A chromatic space is a color space wherein the color information is laid out in as few dimensions as possible, with a separate dimension for luminosity --- the brightness information. Fundamentally, the issue with this is that CCD cameras capture information using RGB values in a spectrum which mixes the chromatic information with the luminosity. Separating this information is the first part of the challenge.

The second part of the challenge is, because we are searching for objects which exhibit certain chromatic characteristics, a statistical model must be developed and applied in order to characterize the objects as such.

The third and final challenge is developing efficient, discrete maths to perform the chromatic space rotation and statistical model efficiently enough such that all the chromatic information about the target object is retained, whilst all irrelevant information is discarded.

While several people have developed algorithms for skin detection, their focus has been squarely on detection rather than retention of information. For color spaces, Hue-Saturation based spaces, such as HSV, have been used due to the clear separation of the chromatic information and luminosity <Zarit and Sigal>. Simpler color spaces, such as Normalized RGB, have been used in video applications due to the demand of continuously processing image frames <Soriano>. As for gathering statistics, histogram thresholding <Soriano and Sigal> and Gaussian models using 2D Gaussians <Terrillon>, 3D Gaussians or multiple Gaussian clusters <Veshnevets> are among the most common models in practice <Shin>, though other models have been used to similar effect, such as the Self-Organizing Map (SOM) in <Brown>'s application. Additionally, practially every application uses double precision numbers in their transformations <Shin, Veshnevets and Terrillon>.

Regardless of the color spaces and statistical models used, all of these algoritms have the same fundamental approach: the image is transformed into some color space, and the statistical model is applied, resulting in a binary image which classifies whether any given pixel is skin. This information is then used as a mask and applied to the original image. This algorithm is fundamentally different to the one used for this project; the primary goal is to preserve information about the skin, as opposed to simply detecting whether a given object is skin or not. We process the image into a new chromatic space, then apply the statistical model, resulting in an image which contains all the chromatic and luminosity information about the skin within it, and information about non-skin areas is lost. 

We are preserving the information in a targeted way, reducing the overall infromation in the image whilst preserving the relevant information about the object we're interested in. This is a clear difference from the more common binary categorization approach, though it is possible to adapt our approach to the same process. However, as it stands, the entire process should be more efficient than the first step in a binary classifier, and should be faster than even moving to an HSV image.


\section{Choosing a Color Space}

As mentioned previously, one of the challenges facing this project is identifying a chromatic space in which the color information and luminosity from the raw RGB image can be separated, thereby simplifying the process of identification of human skin color. To this end, the most widely-used chromatic spaces in practical applications of skin color classification have been evaluated --- HSV, LAB and YCbCr --- all of which have a number of readily-available discrete implementations (\cite{Vezhnevets2003,Zarit1999a,Yang1997a,Brand2000a,Sigal2000a,Chai2000a,Phung2002a}).

Unlike RGB, the HSV color space makes a clear distinction between the chrominance (the "Hue" and "Saturation" channels) and the luminance (the "Value" channel), storing them separately (\cite{Vezhnevets2003,Sigal2000a}). However, the chromatic information in the Hue channel is expressed in polar coordinates, thereby necessitating a coordinate transform when converting from the raw image data. Given the nature of the algorithm outlined herein --- targeted preservation of information, as opposed to classification --- and the mechanical stress application of the algorithm, the computational cost of this transformation is undesirable.

LAB and YCbCr, on the other hand, not only explicity seperate the chrominance and luminance, the information is also expressed linearly (\cite{Vezhnevets2003,Poynton1997,Phung2002a}), thereby allowing for simple and quick conversion to and from the RGB space with no loss of information. However, while they appear to be perfectly suited for the purposes of this project, a number of issues arise when represented discretely. LAB, YCbCr and other such Hue-Saturation spaces all include implicit white-point correction in the readily-available implementations, which distorts the information from the raw image. Also, while the transform is reversible numerically, discrete data types are used in practical computer science applications, and in most implementations the output is the same data type as the input, resulting in some loss of information when the transform is applied. 

Furthermore, based on the skin statistics we have gathered --- which will be described in detail in the next chapter --- the entire region of skin color in the chromatic space can be expressed as a 2D Gaussian. Typically, this is a complex operation, but by aligning the chromatic axes of a Hue-Saturation space with the major and minor axes of the 2D Gaussian, it can be expressed as the product of two 1D Gaussians in each chromatic channel, thereby facilitating the application of the statistics. While it is possible to modify existing implementations of such Hue-Saturation spaces to achieve this, in light of the issues with white-point correction and loss of information due to discrete representation, it was decided that a bespoke chromatic space would be best suited for this project.






For the purposes of this project, a color space dedicated to human skin is necessary. When it comes to pattern recognition, it is beneficial to reduce the number of channels which the algorithm must process, as well as reducing the information density in those channels. It is also hoped that, in an appropriate color space, it will be a simple matter to separate the characteristics associated with saturation of tissue with blood, which changes under mechanical stress from the unchanging pigmentation of the cells (\cite{Stamatas2004}). Color spaces --- when they are normally defined --- are a combination of rotations, scaling and translations, while some (e.g. HSV (\cite{Vezhnevets2003, Zarit1999a} )) also perform a coordinate transform. The color space defined for this project is unusual in that it uses a non-linear redistribution of the values. It should be noted that this is not a pixel-based classifier (\cite{Jones2002}). However, seeing as the color space contains both the statistical information and the skin color information, it can be interpreted in a probabilistic way.

For the sake of comparison, the following four color spaces --- all of which have been used for skin detection (\cite{Vezhnevets2003,Zarit1999a,Yang1997a,Brand2000a,Sigal2000a,Chai2000a,Phung2002a}) --- have been evaluated for the purposes of this project.

\subsection{LAB}\label{sec:LAB}

The LAB color space's perceptual uniformity makes it well-suited for skin detection, as the human eye can perceive practically any change in its values (\cite{Vezhnevets2003,Poynton1997}). However, the Gaussian distribution of the skin is not positioned on the axis, which makes applying the statistics troublesome.

\subsection{RGB}\label{sec:RGB}

The RGB color space, while widely used, is perhaps the most ill-suited color space for skin detection, as it can't make a distinction between chrominance and luminance, as discussed by ~\cite{Vezhnevets2003,Brand2000a}. That said, its ubiquity in cameras --- which are designed to capture images which, to us, appear to be an accurate representation of physical reality despite the opposite being true --- makes it a necessary color space to work in. The advantage to using it is simply the fact that it is a straightforward, raw output from the camera.

\subsection{HSV}\label{sec:HSV}

Unlike RGB, the HSV color space makes a clear distinction between the chrominance (the "Hue" and "Saturation" channels) and the luminance (the "Value" channel), storing them separately (\cite{Vezhnevets2003,Sigal2000a}). This makes HSV a popular choice for skin segmentation, but it's difficult to control any information loss during the transform due to its non-linearity (\cite{Poynton1997}).

\subsection{YCbCr}\label{sec:YCbCr}

This color space bears similarities to the LAB and HSV spaces in that it explicitly separates the chrominance and the luminance. The luminosity (or "luma") channel "Y" is a weighted sum of the RGB values (\cite{Poynton1997,Phung2002a}), while the chrominance channels "Cb" and "Cr" contain the color difference between Y and the red and blue components of RGB, respectively (\cite{Vezhnevets2003}). Also like LAB, it is not oriented such that the Gaussian distribution is situated along the axis. In fact, there is little distinction between the two spaces aside from the orientation in the chromatic portion of the space.

The main issue with using these color spaces is the lack of control over information flow upon transformation and the clear mathematical statement of the transformation in terms of rotation, scaling and translation. Therefore, the construction of a new color space is necessary.

Reasons for LAB: because LXY/LAB spaces all include implicit white-point correction in the readily-available implementatons, so there's no control over that. They all compress the information back into--they all use the same data type, the whole bit about "yes, the transform is reversible, but not really, because we're using discrete data types and most implementations the output is the same data type as the input. You can't fit a box into another box at an angle; the box needs to be big enough for it or it will lose the information. So they all implicitly do that. The second point is because we're using a 2D gaussian which, if we align the major and minor axes, then it's just a 1D gaussian in two of the chrom channels. So applying the stats is better. 

So!

1. Distortion from implicit white-point corrections.

2. Loss of information.

3. To facilitate the application of the stats.



\section{Physiology Study}\label{sec:PhysiologyStudy}

Right, we're going from chemistry to computer science, yeah? This is the point, we're--we're going from physicalto virtual how we want to say it?

The response spectra is how you turn wavelength into RGB. 'cause that's weird. We have a physical quantity, the energy of light, and we have to turn it into three numbers. Weird, huh? That depends upon the measuring device, so if that's an eye, it has a certain response based on the chemistry of the cones, yeah? In our retina. The retina produces that signal, yeah? But there are only three different types of cones which produce that kind of signal--well, we have cones and rods so it's kinda like RGBa, so we detect color and luminosity seperately, but whatever, in terms of color our cones produce RGB. CCDs are based on that and they produce RGB signals, but the absorption-- to do that, we take the wavelength which is generally in nanometers, and a response function, which is dimensionless, it's just the response--y'know the output, one for blue one for green one for red. The red one has, um, for eyes it has a weird bump, and a secondary bump there--the point is that there are a load of different response spectra. So we have a function for R G and B, depending on-- keep it simple, we're not making a big deal about it-- it is whatever it is. And what we have here in mathemativa (wavelength.nb) if you look at mathematica's graphical rep it simulates the red bump of the eye response, and we'll come back to that.

Now, also, we have the absorption spectra. We have some skin, and we have our chromophores. We have hemoglobin, hemo-oxy and three forms of melanin but we have the data for two. But they're good enough; they give us an in. There are other chromophores that can be included which are breakdown products of blood, the stuff which makes bruises green, a variety of chemicals, but we'll restrict ourselves to these major components, the major chromophores in the skin. Now, the chromophores and the ones that contribute to diffusion. Now, chromophores are for absorption and the rest is dispersed (scattered). The point is that we are--we've consistently called it re-emission. ANYWAY, it's what comes out of the surface. So as absorption goes up, remittance goes down. Then there's blood constants, molar mass (very heavy; 64kg/mo), a--nevermind looking at mathematica file now.

If we wanna make this quick, we gotta get the raw data from CCD and process it, but they all look weird; lots of variability, but they're corrected to match the more sensible one, hence all the correction.




The image captured by a CCD camera is designed to fool the human eye. In physical reality, light reaching the sensor spans full spectral range. In the human eye --- and in the camera --- the diversity of chromatic information is lost by representing the spectrum as a combination of three chromatic Gaussians, as can be seen in Figure~\ref{fig:spectrum}.

\begin{figure}[h!]
  \centering
    \includegraphics[width=\textwidth]{Chapter2/Figs/spectrum.eps}
    \caption{Human eye representation of the color spectrum.}  \label{fig:spectrum}
\end{figure}

Human skin is made up of two layers, each with its own optical properties: the surface layer --- the "epidermis" --- and the "dermis," the vascular structure underneath in which the blood vessels, lymphatic vessels and other such fibroblast structures can be found (~\cite{Stamatas2004}). The molecules which make up these layers make an important contribution to skin color. The most significant of these are melanin --- which is found in the epidermis --- and oxy-hemoglobin and deoxy-hemoglobin --- both of which reside in the dermis. Melanin is the main spectral absorber in the epidermis and the greatest contributor to the visual perception of surface skin color; it features a monotonic increase toward short wavelengths on its spectral absorption curve in the visible, approaching linear in the region 600-750 nm (~\cite{Stamatas2004,Kollias1995,Zonios2001}). Oxy-hemoglobin and deoxy-hemoglobin are actually different states of the same molecule --- hemoglobin --- which is found in the dermal blood supply. The molecule's chromophoric state is dependent upon whether it is delivering oxygen molecules to the tissues; it exists as oxy-hemoglobin if it is delivering oxygen and deoxy-hemoglobin if it isn't. The redness of skin in the areas where hemoglobin is found is due to the molecule's absorption of incident light (~\cite{Kollias1995}). It should be noted that each form of hemoglobin has its own characteristic absorption profile; oxy-hemoglobin's maxima on the absorption curve can be found at 415, 540 and 577 nm, whereas deoxy-hemoglobin's are located at 430 and 555 nm.

Seeing as the chromophores' absorption characteristics are expressed as wavelengths, it should be possible to turn a wavelength into a point in a color space. The absorption spectra is a complete description of the material. As such, it can be used to show what the material would look like under any lighting condition. The white light spectrum as perceived by the human eye can be represented using RGB Gaussian waveforms by turning the absorption spectra into a scattering spectra and representing it in terms of the RGB Gaussians. Through the use of an additive process based on Grassman's law, which states that, if the sample color is a combination of two monochromatic colors, then the value found by the observer for each monochromatic base color from the sample color will be the sum of each sample base color of the two monochromatic colors observed separately, we can find points in the R, G and B channels in the RGB color space for the three chromaphores (~\cite{MIHAI2007}).

The equivalent RGB points for the colors of the absorption peaks can be calculated as follows:

\begin{equation}
\begin{array}{cc}
 \{0.6\, -0.0136667 (\lambda -380),0,0.02 (\lambda -380)+0.39\} & \lambda \geq 380.\land \lambda \leq 410. \\
 \{0.19\, -0.00633333 (\lambda -410),0,1\} & \lambda \geq 410.\land \lambda \leq 440. \\
 \left\{0,\frac{\lambda -490}{50}+1,1\right\} & \lambda \geq 440.\land \lambda \leq 490. \\
 \left\{0,1,\frac{510-\lambda }{20}\right\} & \lambda \geq 490.\land \lambda \leq 510. \\
 \left\{\frac{\lambda -580}{70}+1,1,0\right\} & \lambda \geq 510.\land \lambda \leq 580. \\
 \left\{1,\frac{640-\lambda }{60},0\right\} & \lambda \geq 580.\land \lambda \leq 640. \\
 \{1,0,0\} & \lambda \geq 640.\land \lambda \leq 700. \\
 \{0.008125 (780-\lambda )+0.35,0,0\} & \lambda \geq 700.\land \lambda \leq 780. \\
\end{array}
 \\
\end{equation}

If the respective color is shined on a molecule with these properties, it will absorb them. To convert this into a scattering spectrum, spectrum of the incident light must be taken, subtracting the amount that is absorbed. Assuming a uniform line, the process is not unlike flipping the waveform on its Y axis.

\begin{figure}[h!]
  \centering
    \includegraphics[width=0.45\textwidth]{Chapter2/Figs/absOxyHemo.eps}
    \includegraphics[width=0.45\textwidth]{Chapter2/Figs/absDeoxyHemo.eps}
    \includegraphics[width=0.45\textwidth]{Chapter2/Figs/absMelanin.eps}
    \caption{Oxy-hemoglobin, deoxy-hemoglobin and melanin absorption characteristics.}  \label{fig:abs}
\end{figure}

The scattering spectra needs to be represented in a three-function Gaussian basis. We will take the maximum values of these and make a quick transform of the scattering spectra, giving a color value for the molecules and their relative positions in the space. The scattering spectra are presented in Figure~\ref{fig:scat}.

\begin{figure}[h!]
  \centering
    \includegraphics[width=0.45\textwidth]{Chapter2/Figs/scatOxyHemo.eps}
    \includegraphics[width=0.45\textwidth]{Chapter2/Figs/scatDeoxyHemo.eps}
    \includegraphics[width=0.45\textwidth]{Chapter2/Figs/scatMelanin.eps}
    \caption{Oxy-hemoglobin, deoxy-hemoglobin and melanin scattering spectra.}  \label{fig:scat}
\end{figure}

Finally, we perform a color matching operation based on the CIE 1931 standard observer piecewise function to complete the transformation.

\begin{figure}[h!]
  \centering
    \includegraphics[width=\textwidth]{Chapter2/Figs/colorBasis.eps}
    \caption{Final skin chromophore positions.}  \label{fig:colorBasis}
\end{figure}

We have a function for the light source, but we will make the crude assumption of an equal amount of all wavelengths. This is sufficient for the purposes of this project.

%********************************** %First Section  **************************************
%\section{Motivation} %Section - 1.1 





%********************************** %Second Section  *************************************
%\section{Overview of Color Spaces} %Section - 1.2


%********************************** % Third Section  *************************************
%\section{Where does it come from?}  %Section - 1.3 
%\label{section1.3}


