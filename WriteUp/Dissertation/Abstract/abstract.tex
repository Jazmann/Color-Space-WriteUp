% ************************** Thesis Abstract *****************************
% Use `abstract' as an option in the document class to print only the titlepage and the abstract.
\begin{abstract}
Human skin and feature detection are common and widespread applications in computer vision. However, many of the algorithms used by such applications are very resource-intensive and require powerful machines to run them with any reasonable level of performance, if at all. Furthermore, these algorithms tend to focus on detection as opposed to retention of information. The purpose of this project is to design and implement a color space algorithm which not only quickly and accurately detects chromatic skin and related features, but is also efficient enough to run on CCD camera-equipped mobile devices. In the first chapter, we examine techniques and color spaces used in typical skin detection algorithms, exploring the benefits and costs of each, as well as physiological considerations and how CCD devices capture color information, and make the case for a bespoke color space transform. The second chapter describes, in exhaustive detail, the design and construction of the bespoke color space, which stores the chromatic and luminosity information separately; retains color space information in a targeted way, discarding irrelevant data based on a given statistical skin model; and performs the color space transform using integer types, thereby eliminating the need for costly floating-point operations. Chapter three outlines the method of gathering the chromatic skin statistics used to build the bespoke color space described in the second chapter, which is expressed as a 2D Gaussian in the chromatic plane, obtained as a product of two 1D Gaussians, thereby further reducing the data processing cost of the algorithm. The fourth chapter describes the practical application of the bespoke color space based on the theoretical designs in the previous two chapters. In the case of this project, it is a mechanical fingertip-stress detector built using the OpenCV computer vision library in C++ and running on an Apple iPhone 6 Plus with iOS version 10.3.3. In the fifth chapter, we evaluate the results of the application by comparing it with a more typical floating-point implementation, which shows a significant four-to-fivefold improvement over the standard approach.
\end{abstract}
