%*****************************************************************************************
%*********************************** Fourth Chapter **************************************
%*****************************************************************************************

\chapter{Future Work and Discussion}

\ifpdf
    \graphicspath{{Chapter4/Figs/Raster/}{Chapter4/Figs/PDF/}{Chapter4/Figs/}}
\else
    \graphicspath{{Chapter4/Figs/Vector/}{Chapter4/Figs/}}
\fi

The finger press algorithm is really a simple application to demonstrate the viability of detecting blood movement using a standard camera on a mobile device. Many previous authors have dismissed using bespoke color spaces for such applications simply because the color space transform itself is computationally intensive and significantly loses information when applied using standard, off-the-library-shelf color spaces. It's fundamentally difficult to use with the loss of information, with white-out and black-out, and the computational intensity right at the front of the algorithm have meant that such techniques have seen little use.

It is hoped that, with the rigorous optimization of the transform itself --- which is considered in great and incredibly tedious detail in chapter 2 --- that these color spaces can see greater application in the future, although it's recognized that the level of attention that's currently required to make the routine as efficient at the one designed herein will likely be beyond the patience of many practitioners, some of whom are medical professionals and not computer scientists. Also, it is hoped that the work presented here has addressed the concerns about the loss of information associated. This surely must be the case because the color space transform is designed not to lose any information from the RGB color space.

Finally, the use of a Gaussian model for the skin color space has been the source of much debate; some authors insist that the Gaussians be extended to the three-dimensional space, but hopefully it can be seen from this work that a luminosity Gaussian distribution is an artifact due to white-out and black-out, and so is not necessary. The potential gains of using such a luminosity distribution would appear to be gained using the 2-bit Canny edge detection algorithm outlined in~\ref{sec:ImprovedContourDetection}.

The color space algorithm has many possible future applications as an aid to diagnostics, however specialist cameras will always outperform in this area. Also, mechanical stress is not limited to the end of the fingers; knuckles significantly whiten when flexed, so the color space could aid in hand posture work, as would the Canny edge detection methods --- although not presented here during the design of the algorithm --- it was striking how clearly other hand features presented themselves. For instance, it was very tempting to attempt to attach a feature descriptor to the perpendicular creases in the knuckles.

Although not presented here, the majority of the time and effort involved in producing this work went into the furthering of the OpenCV library, which is an open source project hosted on GitHub. The reason this occupied so much of the time is because, at the start of this project, there was no working implementation for iOS. Additionally, the new C++11 standard was released and many of the associated libraries were upgraded, as is so oftenthe case with computer science projects, a significant amount of time had to be dedicated to actually making the code and the libraries work on the selected device appropriately. Probably the most significant contribution this project has made to the library is the extension of the internal data types, which were necessary both for the C++11 standard and the 2-bit and 4-bit types which allowed for the optimization of many of the heavy-duty routines.

As for the finger pressure detector, it could be extended to detect the whole hand and to track the individual digits; with work and some machine learning algorithms, it should be possible to enable the app to measure the degree of pressure, which would be ideally suited to a project such as a paper piano, or perhaps even an augmented reality keyboard.